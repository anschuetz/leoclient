\documentclass{article}

\usepackage{graphicx}
\usepackage{amssymb}
\usepackage[T1]{fontenc}
\usepackage[latin1]{inputenc}
\usepackage{ngerman}
\usepackage{fancybox} 
\usepackage{marvosym}
\usepackage[dvips]{color} 
\usepackage{fancyhdr}
\usepackage{moreverb}
\usepackage[cdot,amssymb,thickqspace]{SIunits}
\setlength{\parindent}{0mm}
\setlength{\parskip}{3mm}
\begin{document}
 

\section{Funktionsprinzip}

Eine virtuelle Festplatte von \verb|VirtualBox| besteht aus 2 Dateien:

\begin{enumerate}
\item Die Basisdatei
\item Die Snapshotdatei
\end{enumerate}

Wird bei ausgeschaltetem virtuellem Rechner die Snapshotdatei ersetzt,
kann man ein anderes, durch die Snapshotdatei festgelegten Image booten.

\verb|leovirtstarter-server| komprimiert und managed die Snapshots auf
dem Server. Evtl. werden die Dateien noch auf weitere Server
gesynct. Dieses Script hat kein GUI.

\verb|leovirtstarter-client| ist ein Programm mikt GUI. 

Es listet die verf�gbaren, komprimierten Snapshot-Dateien des Servers
auf. Es erm�glicht dann, diese Snapshot-Dateien vom Server zu holen,
zu entpacken und zu booten. Dabei werden die Snapshot-Dateien lokal
gecached, um bei wiederholtem booten an demselben Client den
Netzwerkdownload zu vermeiden.

Die Bibliothek \verb|leovirtstarter.pm| wird von beiden Scripten
benutzt.


\section{Konfigurationsdateien}


\subsection{Auf dem Client}

Auf dem Client befindet sich die Konfigurationsdatei:

\verb|/etc/leovirtstarter/leovirtstarter.conf|

In ihr steht als einziger Parameter, wo sich die
Server-Konfigurationsdatei befindet (Pfad im Dateisystem zum
Server-Verzeichnis).


\subsection{Auf dem Server}

Zus�tzlich zu 

\verb|/etc/leovirtstarter/leovirtstarter.conf|

findet sich auf dem Server die Konfigurationsdatei

\verb|/etc/leovirtstarter/leovirtstarter-server.conf|

Beim Aufruf von \verb|leovirtstarter-server| wird diese Datei ins
Image-Verzeichnis kopiert. Das bedeutet, dass nach jeder �nderung in
\verb|leovirtstarter-server.conf| das Serverscript aufgerufen werden
muss.


\subsection{In dem Image-Verzeichnissen auf dem Server}

In der Konfigurationsdatei \verb|image.conf| gibt es folgende
Parameter (alle Optional):

\begin{itemize}
\item \verb|name=Physik Windows mit Versuchen|

  Der Name unter dem das Image in der Auswahlliste erscheinen
  soll. Sinnvoll ist, wenn der Name in Zusammenhang zum
  Verzeichnisnamen des images steht, und nur durch Sonderzeichen,
  Leerzeichen usw. unterscheidet.

\item \verb|maintainer=Loginname| 

  Hier wird der \verb|Loginname| der Person eingetragen die das image
  erstellt hat und weiter pflegen wird. Sie ist auch Ansprechpartner,
  fall das image nicht mehr benutzt wird und zur L�schung
  ansteht. Dazu wird an den Loginamen Mail verschickt.

  Der Name erscheint in Klammern bei der Auswahl des Snapshots.
\end{itemize}


Einschr�nkung auf bestimmte Rechner in \verb|image.conf|. Das Image
wird angezeigt, wenn \textit{eine} der folgenden Bedingungen zutrifft: 

\begin{itemize}
\item \verb|host=hostname1,hostname2,hostname3|

  Die Anzeige dieses Images wird auf die genannten Hosts
  eingeschr�nkt.

\item \verb|room=room1,room2,room3|

  Die Anzeige dieses Images wird auf die Hosts der genannten R�ume
  eingeschr�nkt (Der Raum eines Hosts wird ist die prim�re Gruppe des
  Workstationaccounts mit dem Hostname).
\end{itemize}

Einschr�nkung auf bestimmte Personen, Klassen und Projekte.  Dies kann
�ber die Dateirechte des image-Verzeichnisses eingestellt werden.

Kann man als user den Inhalt dieses Verzeichnisses auflisten, dann
wird das Image angezeigt.



\section{Debugging}

\subsection{Fehlermeldungen des Programmes}


Hier findet sich der output des Programms

\begin{enumerate}
\item Starten auf Konsole: Fehlermeldungen im Konsolenfenster
\item Logdatei ist /tmp/leovirtstarter.log
\end{enumerate}


\subsection{Debugging Optionen}


\begin{itemize}
\item \verb|--host-override new_hostname| nimmt den angegeben Namen
  statt dem, der \verb|hostname| zur�ckliefert.

\item \verb|--room-override new_room| nimmt den angegeben Raum statt
  dem, der durch die prim�re Gruppe des Accounts \verb|hostname|
  zur�ckliefert wird.
\end{itemize}

\end{document}
